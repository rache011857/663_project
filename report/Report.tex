\documentclass{article} % For LaTeX2e
\usepackage{663report,times}
\usepackage{hyperref}
\usepackage{url}
%\documentstyle[nips14submit_09,times,art10]{article} % For LaTeX 2.09


\title{The Multiple-Try Metropolis and its Variations}


\author{
Xu Chen\\
Department of Statistical Science\\
\texttt{xu.chen2@duke.edu} \\
\And
Menglan Jiang\\
Department of Statistical Science\\
\texttt{menglan.jiang@duke.edu}
}

% The \author macro works with any number of authors. There are two commands
% used to separate the names and addresses of multiple authors: \And and \AND.
%
% Using \And between authors leaves it to \LaTeX{} to determine where to break
% the lines. Using \AND forces a linebreak at that point. So, if \LaTeX{}
% puts 3 of 4 authors names on the first line, and the last on the second
% line, try using \AND instead of \And before the third author name.

\newcommand{\fix}{\marginpar{FIX}}
\newcommand{\new}{\marginpar{NEW}}

\nipsfinalcopy % Uncomment for camera-ready version

\begin{document}


\maketitle

\begin{abstract}
Markov chain Monte Carlo (MCMC) has been extensively applied in many complicated computational problems to sample from an arbitrary distribution. The fundamental idea is to generate a Markov chain whose invariant distribution is the target distribution. The traditional Metropolis-Hastings algorithm (MH) based on local search may suffer from slow converging problem since the sampler may get stuck in a local mode. To overcome this difficulty, Jun S. Liu et al. proposed Multiple-try Metropolis (MTM) in 2001. This project will prove the validity of MTM and implement the algorithm and its variations including Conjugate-Gradient Monte Carlo (CGMC) and Langevin-within-MTM on artificial data and real dataset. Comparisons are made to show the superiority of the algorithm over traditional MH algorithms.
\end{abstract}

\section{Introduction}

\subsection{}

\subsection{}

\section{The algorithm and its variations}


\section{Implementation}



\section{Optimization and high performance computing}



\section{Experimental results and comparisons}


\section{Conclusions}


\subsubsection*{References}


\small{
[1] Faming Liang, Chuanhai Liu \& Raymond Carroll (2011) {\it Advanced Markov chain Monte Carlo methods: learning from past samples} vol:714 John Wiley \& Sons

[2] Jun S. Liu (2001) {\it Monte Carlo Strategies in Scientific Computing} Statistics, Springer-Verlag, New York

[3] Jun S. Liu, Faming Liang \& Wing Hung Wong (2001) The Multiple-try method and local optimization in Metropolis Sampling. {\it Journal of the American Statistical Association}, 95:449, pp. 121-134

[4] Radford M. Neal (2011) MCMC using Hamiltonian dynamics. In Steve Brooks et al. (eds.) {\it Handbook of Markov chain Monte Carlo} Chapter 5, Chapman \& Hall/CRC Press
}
\end{document}
